
%%%%%%%%%%%%%%%%%%%%%%%%%%%%%%%%%%%%%%%%%%%%%%%%%%%%%%%%%%%%%%%%%%%%%
%% This is a (brief) model paper using the achemso class
%% The document class accepts keyval options, which should include
%% the target journal and optionally the manuscript type.
%%%%%%%%%%%%%%%%%%%%%%%%%%%%%%%%%%%%%%%%%%%%%%%%%%%%%%%%%%%%%%%%%%%%%
\documentclass[journal=jacsat,manuscript=communication]{achemso}

\setkeys{acs}{articletitle = true}
\SectionsOn
\SectionNumbersOn

%%%%%%%%%%%%%%%%%%%%%%%%%%%%%%%%%%%%%%%%%%%%%%%%%%%%%%%%%%%%%%%%%%%%%
%% Place any additional packages needed here.  Only include packages
%% which are essential, to avoid problems later.
%%%%%%%%%%%%%%%%%%%%%%%%%%%%%%%%%%%%%%%%%%%%%%%%%%%%%%%%%%%%%%%%%%%%%
\usepackage{chemformula} % Formula subscripts using \ch{}
\usepackage{hyperref}
\usepackage{gensymb}
\usepackage[T1]{fontenc} % Use modern font encodings

%%%%%%%%%%%%%%%%%%%%%%%%%%%%%%%%%%%%%%%%%%%%%%%%%%%%%%%%%%%%%%%%%%%%%
%% If issues arise when submitting your manuscript, you may want to
%% un-comment the next line.  This provides information on the
%% version of every file you have used.
%%%%%%%%%%%%%%%%%%%%%%%%%%%%%%%%%%%%%%%%%%%%%%%%%%%%%%%%%%%%%%%%%%%%%
%%\listfiles

%%%%%%%%%%%%%%%%%%%%%%%%%%%%%%%%%%%%%%%%%%%%%%%%%%%%%%%%%%%%%%%%%%%%%
%% Place any additional macros here.  Please use \newcommand* where
%% possible, and avoid layout-changing macros (which are not used
%% when typesetting).
%%%%%%%%%%%%%%%%%%%%%%%%%%%%%%%%%%%%%%%%%%%%%%%%%%%%%%%%%%%%%%%%%%%%%
% \newcommand*\mycommand[1]{\texttt{\emph{#1}}}
\newcommand*\compound{\ch{Co(phen)2(NCS)2} }
%%%%%%%%%%%%%%%%%%%%%%%%%%%%%%%%%%%%%%%%%%%%%%%%%%%%%%%%%%%%%%%%%%%%%
%% Meta-data block
%% ---------------
%% Each author should be given as a separate \author command.
%%
%% Corresponding authors should have an e-mail given after the author
%% name as an \email command. Phone and fax numbers can be given
%% using \phone and \fax, respectively; this information is optional.
%%
%% The affiliation of authors is given after the authors; each
%% \affiliation command applies to all preceding authors not already
%% assigned an affiliation.
%%
%% The affiliation takes an option argument for the short name.  This
%% will typically be something like "University of Somewhere".
%%
%% The \altaffiliation macro should be used for new address, etc.
%% On the other hand, \alsoaffiliation is used on a per author basis
%% when authors are associated with multiple institutions.
%%%%%%%%%%%%%%%%%%%%%%%%%%%%%%%%%%%%%%%%%%%%%%%%%%%%%%%%%%%%%%%%%%%%%
\author{Adam Keim}
\email{a_keim@coloradocollege.edu}

\author{Amanda Bowman}
\email{abowman@coloradocollege.edu}
\affiliation[Colorado College]{Department of Chemistry, Colorado College, Colorado Springs}


%%%%%%%%%%%%%%%%%%%%%%%%%%%%%%%%%%%%%%%%%%%%%%%%%%%%%%%%%%%%%%%%%%%%%
%% The document title should be given as usual. Some journals require
%% a running title from the author: this should be supplied as an
%% optional argument to \title.
%%%%%%%%%%%%%%%%%%%%%%%%%%%%%%%%%%%%%%%%%%%%%%%%%%%%%%%%%%%%%%%%%%%%%
\title[An \textsf{achemso} demo]
  {An investigation of the potential spin-crossover properties of \ch{Co(phen)2(NCS)2}}

%%%%%%%%%%%%%%%%%%%%%%%%%%%%%%%%%%%%%%%%%%%%%%%%%%%%%%%%%%%%%%%%%%%%%
%% Some journals require a list of abbreviations or keywords to be
%% supplied. These should be set up here, and will be printed after
%% the title and author information, if needed.
%%%%%%%%%%%%%%%%%%%%%%%%%%%%%%%%%%%%%%%%%%%%%%%%%%%%%%%%%%%%%%%%%%%%%
\abbreviations{IR,NMR,UV}
\keywords{American Chemical Society, \LaTeX}

%%%%%%%%%%%%%%%%%%%%%%%%%%%%%%%%%%%%%%%%%%%%%%%%%%%%%%%%%%%%%%%%%%%%%
%% The manuscript does not need to include \maketitle, which is
%% executed automatically.
%%%%%%%%%%%%%%%%%%%%%%%%%%%%%%%%%%%%%%%%%%%%%%%%%%%%%%%%%%%%%%%%%%%%%
\begin{document}


%%%%%%%%%%%%%%%%%%%%%%%%%%%%%%%%%%%%%%%%%%%%%%%%%%%%%%%%%%%%%%%%%%%%%
%% The abstract environment will automatically gobble the contents
%% if an abstract is not used by the target journal.
%%%%%%%%%%%%%%%%%%%%%%%%%%%%%%%%%%%%%%%%%%%%%%%%%%%%%%%%%%%%%%%%%%%%%

\begin{abstract}
In this project, we synthesized a cobalt-containing complex and attempted to characterize its electronic structure using a variety of experimental and computational techniques. Our TGA+electromagnet findings yielded promising initial results about the compound's spin crossover properties, but the compound may have degraded, so further research is needed to confirm these properties. We plan to collaborate with other institutions that have SQUID magnetometers to further contribute to the growing body of research on spin crossover materials and their properties.
\end{abstract}

%%%%%%%%%%%%%%%%%%%%%%%%%%%%%%%%%%%%%%%%%%%%%%%%%%%%%%%%%%%%%%%%%%%%%
%% Start the main part of the manuscript here.
%%%%%%%%%%%%%%%%%%%%%%%%%%%%%%%%%%%%%%%%%%%%%%%%%%%%%%%%%%%%%%%%%%%%%
\section{Introduction}
Spin crossover is a property exhibited by certain transition metal complexes where the electronic properties of the material change in response to external stimuli, such as temperature, pressure, or light. The phenomenon is caused by the splitting of the d orbital of the metal, which allows for both high-spin and low-spin configurations.  In the high-spin configuration, the complex adopts a configuration with more unpaired electrons because it is thermodynamically favorable for the electron to move up to the next orbital rather than couple with another electron in the lower orbital.  In the low-spin configuration, the energy gap is large enough that this "jump" is not thermodynamically favorable.

In this project, I synthesized and attempted to characterize the electronic structure of \compound in an attempt to further understand its possible spin crossover properties.  This compound potentially exhibits spin crossover properties, since the \ch{Fe} analog (\ch{Fe(phen)2(NCS)2}) has been demonstrated to exhibit the desired property.  To characterize the spin crossover properties of the cobalt compound, I attempted to use IR, Raman, and TGA techniques, where TGA proved most useful, and would ideally use SQUID magnetometry.  

I also attempted to use DFT techniques to computationally predict the electronic structure of both the high- and low-spin forms of \compound.  This was done in an attempt to be able to compare experimental and computational data to better understand the properties of the compound.  By better understanding the properties of the spin crossover of this particular compound, we hope to further our understanding of the broader mechanism by which spin crossover happens. 

Because spin-crossover often exhibits a hysteresis curve (the material can exist in either state given certain conditions depending on its past), these materials can have the ability to store information, leading to interesting applications for non-NAND memory devices.  Also, since these compounds are able to respond to external stimulus, there is research into their applications as sensors of heat/light.


\section{Materials \& Methods}

\subsection{Synthesis}
The synthesis of \compound was carried out according to the procedure described by \citeauthor{Synthesis}. 0.4217 g of cobalt (II) sulfate hexahydrate were dissolved in approximately 50 mL of water. Then, 0.2933 g of \ch{KSCN} were added to the solution, followed by approximately 25 mL of water to ensure that the thermometer was immersed in the solution. A solution of 0.5417 g of 1,10-phenanthroline in 50 mL of water was added dropwise to the reaction vessel, resulting in the formation of a hot pink precipitate. The precipitate was vacuum filtered after 5 minutes to ensure complete precipitation, and then washed with water and ethanol. The resulting compound was dried under vacuum overnight and had a final dry mass of 0.641 g, yielding 80\%.  
\subsection{Characterization Methods}
\subsubsection{Melting Point}
Melting point characterization was attempted on an SRS Digimelt MPA160.
\subsubsection{IR}
IR Spectra of the compound were taken on a Bruker Alpha in the solid state.
\subsubsection{Raman}
We used a Renishaw Raman microscope to take spectra of the compound, with a 785nm laser and 1200l/mm grating.  The compound was placed onto a glass slide in the solid state.  This slide was then placed in a temperature-controlled heating stage.
\subsubsection{TGA}
All TGA was carried out on a TA Instruments TGA5500.  Initial decomposition experiments used an alumina ceramic pan, and all subsequent experiments used aluminum pans with stainless steel bales.  All samples were put directly in the pans.  
\subsubsection{SQUID}
Our initial contact for SQUID magnetometry had a temporarily non-functional instrument.  SQUID has proven to be one of the most powerful techniques for characterizing spin crossover compounds, and we are in touch with a number of other institutions in the area that may be able to help us accomplish SQUID magnetometry of our desired compound.

\subsection{Computational Investigations}
Computational studies were carried out on a HPC cluster at Colorado College.  Initial calculations were performed using ORCA 4.2.1 and later calculations wtih ORCA 5.0.3.  Initial pre-optimization XYZ coordinates were calculated using Avogadro on Mac OS X.  The calculations used a B3LYP basis set.  Full information on alternate basis sets, convergence criteria, etc. can be found in the supplemental information.  We were able to successfully complete geometry optimization and attempted vibrational frequency calculation of Raman and IR spectra.

\section{Results \& Discussion}
\subsection{Experimental}
This compound proved elusive to characterization techniques.  Initial melting point measurements were unsuccessful, given that the instrument only goes up to 260 \degree C.  

TGA proved to be the most useful technique we had accessible to us, and was the only confirmation of possible spin crossover properties that we were able to see during the short time available to carry out this research.  Through TGA, we were able to see that the compound began degrading around 380 \degree C, as shown in Figure \ref{fig:decomposition}. We were also able to use the TGA to verify possible spin crossover properties (The effective mass with the electromagnet on increased significantly with temperature).  

The main issue that we ran into in attempts to characterize the compound was that we were unable to replicate our original TGA results.  TGA results run shortly after the compound was synthesized (1-2 days) showed an increase in mass with an increase in temperature, but after waiting a couple more days (leaving the compound exposed to air overnight) these results no longer appeared.  

\begin{figure}
  \caption{Decomposition of \compound in TGA}
  \label{fig:decomposition}
\end{figure}

\begin{figure}
	\caption{Possible spin crossover of \compound in TGA}
	\label{fig:TGA_crossover}
\end{figure}

\subsection{Computational}

Our computational studies have yielded limited useful information at this stage. We initially attempted to use ORCA 4.2.1 to perform density functional theory (DFT) calculations, but were unable to obtain convergence for the self-consistent field (SCF) calculations necessary for vibrational frequency calculations. We eventually were able to obtain converged SCF calculations for structure (not vibrational calculations), but our results were still limited by the fact that we did not take into account the solid-state crystalline properties of our compound, which can affect its vibrational spectra.

To address this issue, we would need to perform additional DFT calculations using a crystal-corrected method that accounts for the solid-state nature of the compound. This would allow us to obtain more accurate predictions of the vibrational frequencies of the compound in both its high-spin and low-spin configurations. We could also calculate the energy differences between the high- and low-spin states, which would provide insight into the factors that drive the spin crossover process.

Overall, our computational studies suggest that the cobalt-containing complex studied in this project may exhibit spin crossover properties, but further research is needed to confirm these results experimentally and to better understand the mechanisms underlying the spin crossover process. Additional computational studies that incorporate the effects of temperature and other external stimuli on the electronic structure of the compound could also provide valuable insights into its potential applications.
\subsection{Discussion}


One of the main challenges we faced in attempting to characterize the compound was the difficulty in reproducing our initial TGA results. When the TGA experiments were performed shortly after the compound was synthesized (1-2 days), we observed an increase in mass with an increase in temperature, which is consistent with spin crossover behavior. However, when we waited a few more days (leaving the compound exposed to air overnight) before performing the experiments again, these results were no longer observed. This may indicate that the compound degraded or changed in some way over time, which could have affected its properties. 

Currently, the main limitation in our computational studies is that we were unable to obtain converged SCF calculations for vibrational frequency calculations using ORCA 4.2.1. Once we are able to obtain convergence in these calculations (hopefully updating ORCA helps us achieve this), our results will still be incorrect because we are not taking into account the solid-state properties of the material. 





\section{Conclusion \& Further Directions}
\subsection{Conclusion}
In conclusion, our results provide initial evidence for the potential spin crossover properties of the cobalt-containing complex \compound, but further research is needed to confirm these properties and to better understand the mechanisms underlying the spin crossover process. Our study highlights the challenges and limitations of the techniques used to characterize spin crossover materials and the need for further research using more advanced techniques and computational methods, as well as the difficulties inherent in working with possibly air-sensitive compounds.

\subsection{Further Directions for this research}
\subsubsection{Experimental}

Attempting to reproduce the initial TGA results to confirm the presence of spin crossover properties in the cobalt-containing complex \compound.

Investigating the effects of temperature and other external stimuli on the compound using TGA or other characterization techniques.

Collaborating with other researchers who have expertise in the characterization and study of spin crossover materials to benefit from their knowledge and experience, specifically with SQUID equipment.

\subsubsection{Computational}
Using more advanced computational methods, such as crystal-corrected density functional theory (DFT) calculations, to better understand the mechanisms underlying the spin crossover process in \compound.

Investigating the effects of temperature and other external stimuli on the electronic structure of \compound using computational methods.



%%%%%%%%%%%%%%%%%%%%%%%%%%%%%%%%%%%%%%%%%%%%%%%%%%%%%%%%%%%%%%%%%%%%%
%% The "Acknowledgement" section can be given in all manuscript
%% classes.  This should be given within the "acknowledgement"
%% environment, which will make the correct section or running title.
%%%%%%%%%%%%%%%%%%%%%%%%%%%%%%%%%%%%%%%%%%%%%%%%%%%%%%%%%%%%%%%%%%%%%
\begin{acknowledgement}
The authors thank Dan Ellsworth in the computer science department for his help running the computational calculations on multi-node clusters.  

\end{acknowledgement}

%%%%%%%%%%%%%%%%%%%%%%%%%%%%%%%%%%%%%%%%%%%%%%%%%%%%%%%%%%%%%%%%%%%%%
%% The same is true for Supporting Information, which should use the
%% suppinfo environment.
%%%%%%%%%%%%%%%%%%%%%%%%%%%%%%%%%%%%%%%%%%%%%%%%%%%%%%%%%%%%%%%%%%%%%
\begin{suppinfo}

All data from all instrument runs can be found at the below URL.
\begin{itemize}
  \item \url{https://github.com/adam-keim/CH302}
\end{itemize}

\end{suppinfo}

%%%%%%%%%%%%%%%%%%%%%%%%%%%%%%%%%%%%%%%%%%%%%%%%%%%%%%%%%%%%%%%%%%%%%
%% The appropriate \bibliography command should be placed here.
%% Notice that the class file automatically sets \bibliographystyle
%% and also names the section correctly.
%%%%%%%%%%%%%%%%%%%%%%%%%%%%%%%%%%%%%%%%%%%%%%%%%%%%%%%%%%%%%%%%%%%%%
% \bibliography{bib}

\end{document}
